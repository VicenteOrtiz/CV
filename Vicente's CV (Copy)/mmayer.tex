%%%%%%%%%%%%%%%%%
% This is an example CV created using altacv.cls (v1.1.5, 1 December 2018) written by
% LianTze Lim (liantze@gmail.com), based on the
% Cv created by BusinessInsider at http://www.businessinsider.my/a-sample-resume-for-marissa-mayer-2016-7/?r=US&IR=T
%
%% It may be distributed and/or modified under the
%% conditions of the LaTeX Project Public License, either version 1.3
%% of this license or (at your option) any later version.
%% The latest version of this license is in
%%    http://www.latex-project.org/lppl.txt
%% and version 1.3 or later is part of all distributions of LaTeX
%% version 2003/12/01 or later.
%%%%%%%%%%%%%%%%

%% If you are using \orcid or academicons
%% icons, make sure you have the academicons
%% option here, and compile with XeLaTeX
%% or LuaLaTeX.
% \documentclass[10pt,a4paper,academicons]{altacv}

%% Use the "normalphoto" option if you want a normal photo instead of cropped to a circle
% \documentclass[10pt,a4paper,normalphoto]{altacv}

\documentclass[10pt,a4paper,ragged2e]{altacv}

%% AltaCV uses the fontawesome and academicon fonts
%% and packages.
%% See texdoc.net/pkg/fontawecome and http://texdoc.net/pkg/academicons for full list of symbols. You MUST compile with XeLaTeX or LuaLaTeX if you want to use academicons.

% Change the page layout if you need to
\geometry{left=2cm,right=10cm,marginparwidth=6.8cm,marginparsep=1.2cm,top=1.25cm,bottom=1.25cm}

% Change the font if you want to, depending on whether
% you're using pdflatex or xelatex/lualatex
\ifxetexorluatex
  % If using xelatex or lualatex:
  \setmainfont{Carlito}
\else
  % If using pdflatex:
  \usepackage[utf8]{inputenc}
  \usepackage[T1]{fontenc}
  \usepackage[default]{lato}
\fi

% Change the colours if you want to
\definecolor{VividPurple}{HTML}{000000}
\definecolor{SlateGrey}{HTML}{2E2E2E}
\definecolor{LightGrey}{HTML}{2E2E2E}
\colorlet{heading}{VividPurple}
\colorlet{accent}{VividPurple}
\colorlet{emphasis}{SlateGrey}
\colorlet{body}{LightGrey}

% Change the bullets for itemize and rating marker
% for \cvskill if you want to
\renewcommand{\itemmarker}{{\small\textbullet}}
\renewcommand{\ratingmarker}{\faCircle}

%% sample.bib contains your publications
\addbibresource{sample.bib}

\begin{document}
\name{Vicente I. Ortiz Arancibia}
\tagline{Software Engineer}
% Cropped to square from https://en.wikipedia.org/wiki/Marissa_Mayer#/media/File:Marissa_Mayer_May_2014_(cropped).jpg, CC-BY 2.0
%\photo{3.3cm}{profile.jpg}
\personalinfo{%
  % Not all of these are required!
  % You can add your own with \printinfo{symbol}{detail}
  \email{vicente.ortiz@usach.cl}
%   \phone{000-00-0000}
%  \mailaddress{Address, Street, 00000 County}
  \location{Santiago, Chile}
%  \homepage{marissamayr.tumblr.com/}
%  \twitter{@marissamayer}
  \linkedin{linkedin.com/in/vicente-ortiz-arancibia-18769a147/}
   \github{github.com/VicenteOrtiz/} % I'm just making this up though.
%   \orcid{orcid.org/0000-0000-0000-0000} % Obviously making this up too. If you want to use this field (and also other academicons symbols), add "academicons" option to \documentclass{altacv}
}

%% Make the header extend all the way to the right, if you want.
\begin{fullwidth}
\makecvheader
\end{fullwidth}

%% Depending on your tastes, you may want to make fonts of itemize environments slightly smaller
\AtBeginEnvironment{itemize}{\small}

%% Provide the file name containing the sidebar contents as an optional parameter to \cvsection.
%% You can always just use \marginpar{...} if you do
%% not need to align the top of the contents to any
%% \cvsection title in the "main" bar.
\cvsection[page1sidebar]{Experience}

\cvevent{Lions Up}{USACH ( Universidad de Santiago de Chile )}{Marzo 2017 -- Junio 2017}{Santiago, Chile}
\begin{itemize}
\item Programa de innovación y emprendimiento que apunta a resolver desafíos del hoy en día propuestos por diferentes organizaciones, tanto publica como privada.
\smallskip
\item Trabaje con un equipo multidisciplinario en el diseño, desarrollo y prototipado en tiempo real de una solución que buscaba contribuir a una economía circular.
\smallskip
\item Desarrolle experiencia y conocimiento en design thinking, trabajo en equipo, prototipado y análisis del problema y usuario.
\end{itemize}

\divider

\cvevent{StartUp Chile}{CORFO}{Agosto 2017 -- Diciembre 2017}{Santiago, Chile}
\begin{itemize}
\item Participante del programa SUPBRIDEGE de StartUp Chile, auspiciado por CORFO.
\smallskip
\item Desarrolle experiencia y conocimiento sobre análisis de usuario, modelo de negocio, investigación de mercado, entre otros conocimientos varios sobre la innovación y emprendimiento.
\item Trabaje en conjunto con otras startups, aprendiendo de ellas, y generando contactos.
\end{itemize}

\divider

\cvevent{AngelHack-BCI Hackathon}{BCI Headquarters}{Junio 2018}{Santiago, Chile}
\begin{itemize}
\item Hackaton del sector público y privado. Busca resolver problemas globales y del mundo bancario mediante una competencia con cerca de 50 equipos en 48 horas.
\smallskip
\item Desarrollamos una aplicación web usando Bootstrap, Laravel y MySQL, que resolvía un problema de automatización que tenía el banco BCI en un nuevo proyecto que estaban desarrollando llamado "Social Store".
\item Trabaje con un equipo multidisciplinario, entre ellos: desarrolladores, diseñadores, entre otros. Tuve reuniones con la gerencia de Marketing, Tecnología y BCI Nace para tener una buena toma de requerimientos.
\end{itemize}

\divider

\cvevent{Despega USACH}{USACH ( Universidad de Santiago de Chile )}{Julio 2018 -- Diciembre 2018}{Santiago, Chile}
\begin{itemize}
\item Programa de emprendimiento que busca promover la cultura de innovación y emprendimiento apadrinando proyectos con base científica tecnológica. La idea es resolver problemas, mejorar procesos, generar productos y servicios innovadores.
\smallskip
\item Trabaje en conjunto con desarrolladores trabajando en una aplicación móvil y aplicación web cuya funcionalidad primaria e innovadora era identificar a un perro solo sacandole una foto a la nariz del perro usando redes neuronales.
\item Asistí a cursos sobre emprendimiento, modelo de negocios, investigación de mercado y presentación efectiva
\end{itemize}

%\divider

\cvsection{Logros}
\smallskip
\begin{itemize}
\item Finalista en el programa Despega 2018 de la Universidad de Santiago
\smallskip
\item Debido a todos los programas y concursos que he participado a lo largo de mi vida universitaria, he desarrollado habilidades de innovación, emprendimiento y liderazgo.
\smallskip
\item Califiqué a Jump Chile. Un programa de innovación enfocado en estudiantes universitarios, respaldado por la Universidad Católica de Chile y auspiciado por CORFO.
\end{itemize}

\cvsection{SKILLS}

\cvskill{Design Thinking}{4}
%\divider
\cvskill{Liderazgo}{4}
%\divider
\cvskill{Análisis de Problema y Usuario}{4}
%\divider

\cvskill{C, Python, Java, C\#}{3}
% \divider
\cvskill{Desarrollo web, Bases de dato, SpringBoot}{3}
% \divider
%\cvskill{German}{3}


\cvsection{Educación / Cursos}
\cvevent{Ingeniería Informática}{Universidad de Santiago, Santiago, Chile}{ Junio 2017 -- Presente(Graduación esperada para fines de 2020)}{}
%\divider
\cvevent{Essentials of Entrepreneurship Behavior}{University of Waterloo, Waterloo, Canada}{ Junio 2017 -- Julio 2017}{}
% \divider

% \cvevent{Product Engineer}{Google}{23 June 1999 -- 2001}{Palo Alto, CA}

% \begin{itemize}
% \item Joined the company as employe \#20 and female employee \#1
% \item Developed targeted advertisement in order to use user's search queries and show them related ads
% \end{itemize}

%\cvsection{A Day of My Life}

% Adapted from @Jake's answer from http://tex.stackexchange.com/a/82729/226
% \wheelchart{outer radius}{inner radius}{
% comma-separated list of value/text width/color/detail}
% Some ad-hoc tweaking to adjust the labels so that they don't overlap
% \wheelchart{1.5cm}{0.5cm}{%
%   10/10em/accent!30/Sleeping \& dreaming about work,
%   25/9em/accent!60/Public resolving issues with Yahoo!\ investors,
%   5/13em/accent!10/\footnotesize\\[1ex]New York \& San Francisco Ballet Jawbone board member,
%   20/15em/accent!40/Spending time with family,
%   5/8em/accent!20/\footnotesize Business development for Yahoo!\ after the Verizon acquisition,
%   30/9em/accent/Showing Yahoo!\ employees that their work has meaning,
%   5/8em/accent!20/Baking cupcakes
% }

\clearpage

% \cvsection[page2sidebar]{Publications}

\nocite{*}

% \printbibliography[heading=pubtype,title={\printinfo{\faBook}{Books}},type=book]

% \divider

% \printbibliography[heading=pubtype,title={\printinfo{\faFileTextO}{Journal Articles}}, type=article]

% \divider

% \printbibliography[heading=pubtype,title={\printinfo{\faGroup}{Conference Proceedings}},type=inproceedings]

% %% If the NEXT page doesn't start with a \cvsection but you'd
% %% still like to add a sidebar, then use this command on THIS
% %% page to add it. The optional argument lets you pull up the
% %% sidebar a bit so that it looks aligned with the top of the
% %% main column.
% % \addnextpagesidebar[-1ex]{page3sidebar}


\end{document}
