
\cvsection{Honores \& Premios}
\begin{itemize}
    \item Ganador Lions Up 2017 USACH.
    \item Ganador Hackathon AngelHack-BCI en el desafío de redes sociales de BCI.
\end{itemize}
% \cvachievement{\faTrophy}{}{Received accolades at Atos for Best Performance in team.}
% \cvachievement{\faTrophy}{}{Received Best Debut Award at Atos. }
% %\divider
% \cvachievement{\faInstitution}{}{Won 2nd Consolation Prize for paper presented on Cognitive Radio Networks.}
% %\divider
% \cvachievement{\faGraduationCap}{}{Got Selected in "Exclusive Scholar Program" during undergrad.}
% %\divider
% \cvachievement{\faDollar}{}{Awarded with Narotam Sekhsaria Foundation Scholarship}
%\cvsection{Strengths}

%\cvtag{Hard-working (18/24)} 
%\cvtag{Persuasive}
%\cvtag{Motivator \& Leader}

%\divider\smallskip

%\cvtag{UX}
%\cvtag{Mobile Devices \& Applications}
%\cvtag{Product Management \& Marketing}


%\divider

%\cvevent{B.S.\ in Symbolic Systems}{Stanford University}{Sept 1993 -- June 1997}{}

\cvsection{Proyectos}
\cvproject{Vo'Confia}
\begin{itemize}
\item Fundé un negocio que vende preservativos de alta calidad y a muy bajo precio.
\item La idea es mejorar la educación sexual en la población, empezando primero en mi Universidad y después expandiendo el negocio a otras universidades.
\end{itemize}
\smallskip
\cvproject{Tutoria2}
\begin{itemize}
\item Primer proyecto de ámbito tecnológico. Desarrolle una aplicación que permite a estudiantes universitarios ofrecer clases, tutorías y apoyo a estudiantes de colegio a cambio de una remuneración.
\end{itemize}
\smallskip
\cvproject{Framech}
\begin{itemize}
\item Este proyecto es el que desarrolle en Lions up. El proyecto consistió en crear una plataforma que busca maximizar las utilidades que se pueden generar los desechos de oficina de pequeñas y medianas empresas.
\item El resultado fue que creamos un modelo de economía circular, donde las compañías nos proveían su desechos de oficina, nosotros los vendíamos en grandes cantidades a compañías de reciclaje, y en retorno, nosotros devolvíamos artículos de oficina a las pequeñas y medianas empresas que trabajaban con nosotros.
\item Trabajamos con una empresa llamada Everis, teniendo reuniones mensuales con ellos en sus oficinas.
\end{itemize}
\smallskip
\cvproject{TREWA}
\begin{itemize}
\item Proyecto con el que participé en Despega USACH. El proyecto fue acerca del desarrollo de una aplicación tanto móvil como web cuyo propósito era ayudar en el proceso de buscar e identificar una mascota.
\item Tenía muchas opciones para ayudar en el proceso. Como identificación con código QR, Fotografía y NFC
\item La función mas importante era la posibilidad de identificar a un perro solo con la fotografía de su nariz. Usamos redes neuronales para analizar e identificar los diferentes patrones que tienen los perros en su nariz.
\end{itemize}
\smallskip
\cvproject{DocQr}
\begin{itemize}
\item El objetivo es digitalizar las recetas médicas para ayudar y agilizar la visita de la tercera edad a las consultas médicas.
\vspace{2cm}
\item La idea principal es facilitar el proceso de entrevista del doctor al paciente, donde es necesario saber que medicamentos está tomando el paciente, donde y cuando fueron sus últimas visitas, además de otras informaciones importantes.
\smallskip
\item Actualmente se encuentra en desarrollo la plataforma que agilizará el proceso.
\end{itemize}

%\divider
\smallskip

%\cvproject{BCI-SocialStore}
%\begin{itemize}
%\item Developed a model to learn regular patterns from sensor data and detect unusual pattern.
%\end{itemize}
\smallskip

\cvproject{}

